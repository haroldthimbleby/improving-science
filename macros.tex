% due to finding out that PDF doesn't work everywhere, here's where we define the image type we're using for imaged created by the Mathematica notebooks
\def\imageFileSuffix{jpg}

\usepackage[hyphens]{url}
\usepackage{hyperref}
\usepackage{xcolor}
\usepackage{longtable}
\usepackage{cite}

\renewcommand\thesubsection{\thesection.\alph{subsection}}
\renewcommand\thesubsubsection{\thesection.\alph{subsection}.\arabic{subsubsection}}

\def\httpURL#1{\href{http://#1}{\textcolor{blue}{#1}}}
\def\LONGhttpURL#1#2{\href{http://#1}{\textcolor{blue}{#2}}}

\makeatletter
    \def\do@url@hyp{\do/}
    \usepackage{moreverb}
\makeatother

\DeclareUrlCommand\doi{\def\UrlLeft{{\textrm{DOI}\ } }}
\DeclareUrlCommand\url{\def\UrlLeft{{\textrm{URL}\ } }}
\DeclareUrlCommand\midurl{\def\UrlLeft{}\allowbreak }
	
\renewcommand{\thefootnote}{\arabic{footnote}}

\def\mytitle{Improving science that uses code}
\def\supplement{Supplemental Material}

\def\citeeg#1{[e.g., \citenum{#1}]}

% calculate rounded percentages
\newcount \t
\newcount \tu
\def\pc#1#2{\t = #1%
\multiply \t by 100000%
\divide \t by #2% ?? 0 times percentage
\tu = \t
\divide \t by 1000% percentage, but no idea of remainder
\multiply \t by 1000% 1000 times percentage, with 00 as bottom digits
\advance \tu by -\t% bottom two digits
\divide \t by 1000\relax%
% round to nearest integer
% if ending in > .5 round up
% if ending in exactly .5 round towards nearest even number
\ifnum \tu > 500 % round up
	\advance \t by 1 
\else
	\ifnum \tu = 500 
		\ifodd \t % it's odd, so round up to even number 
			\advance \t by 1
		\else % leave rounded to even number 	
		\fi
	\fi
\fi
\the\t\%}

% code to make digits into blocks of 3 (as in 1000 -> 1 000)
\def\xBlockSep{\hskip 0.3em} % International Standards are to use a small space. UK typically uses a comma
% NOTE currently, the few decimal numbers (eg 0.000 5) use hskip explicitly, since defining \xBlockSep as a comma would definitely be wrong for decimals
\newcount\xdigitBlockn
\newcount\xdigitBlocktmp
\newcount\xdigitBlocktmq

\def\xdigitBlock#1{%
        \xdigitBlockn = #1 \relax
        \ifnum \xdigitBlockn < 0
                $-$%
                \xdigitBlockn = -\xdigitBlockn
        \fi
        \let \xdigitblockinternalmacro \empty
        \loop 
           \ifnum \xdigitBlockn > 999
                \xdigitBlocktmp = \xdigitBlockn
                \divide \xdigitBlockn by 1000
                
                % following line rewritten explicitly for old TeX 
                %\xdigitBlocktmp = \numexpr \xdigitBlocktmp - 1000*\xdigitBlockn \relax           
                \xdigitBlocktmq = \xdigitBlockn \relax
                \multiply \xdigitBlocktmq by 1000
                \advance \xdigitBlocktmp by -\xdigitBlocktmq
                
                \edef \xdigitblockinternalmacro{%
                	\xBlockSep
					\ifnum \xdigitBlocktmp < 100 0\ifnum \xdigitBlocktmp < 10 0\fi \fi 
					\the\xdigitBlocktmp
					\xdigitblockinternalmacro
				}%
        \repeat 
        \the\xdigitBlockn
        \xdigitblockinternalmacro
}

\def\digitBlock#1{\hbox{\xdigitBlock{#1}}}

% simple macro to use a number register to pluralize (add an s) to words....
% \plural{\t}{fred} => \the\t\ freds if plural, or => one fred if singular
\def\plural#1#2{\ifnum #1=1 
	one #2%
\else
	\the#1\ #2s%
\fi}

% initialiseBibliography <title> <starting number> <introductory text>
\newcount \bibciten \bibciten=0
\def\initialiseBibliography#1#2#3{% title, starting citation number, an initial paragraph
	\global\def\refname{\arabic{section} #1}
    \global\def\unnumberedrefname{#1}
    \global\bibciten=#2
    \global\def\startBibliography{#3}
}

\def\bibskip{\vskip 1ex} % the gap between bib items generated by data.js

% read in any constants defined from the JSON or Git data
\input generated/constants.tex
\input generated/metadata.tex
\input generated/covidsimSummary.tex
\input generated/clone-date.tex

\newcount \pubdelayinmonths
\pubdelayinmonths=\cloneyear
\advance \pubdelayinmonths by -2019 % submitted in 2019
\multiply \pubdelayinmonths by 12
\advance \pubdelayinmonths by -6 % submitted in June
\advance \pubdelayinmonths by \clonemonth

\def\monthname#1{\ifcase#1 \or 
January\or 
February\or 
March\or 
April\or 
May\or 
June\or 
July\or 
August\or 
September\or 
October\or 
November\or 
December\fi}

\def\numbertoname#1{\ifcase #1 zero\or one\or two\or three\or four\or five\or six\or seven\or eight\or nine\else #1\fi}
\def\Numbertoname#1{\ifcase #1 Zero\or One\or Two\or Three\or Four\or Five\or Six\or Seven\or Eight\or Nine\else #1\fi}

\def\clonewhen{\monthname{\clonemonth} \cloneyear}

\def\RAPstar{RAP{\tt +}}
\def\RAPstars{RAPs{\tt +}}
\def\RAPstarp#1{RAP{\tt +}#1}
\def\italicRAPstar{RAP{\tt +}}
