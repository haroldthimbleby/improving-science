% Edit makefile to change the selected make command choices, as follows:
% 1 - make all
% 2 - make archive
% 3 - make check-same
% 4 - make check-update
% 5 - make check-versions
% 6 - make data
% 7 - make expand
% 8 - make git-prep
% 9 - make help
% 10 - make help-brief
% 11 - make mathematica
% 12 - make mathematica-open
% 13 - make one-file
% 14 - make pdf
% 15 - make push
% 16 - make readme
% 17 - make really-tidyup
% 18 - make tidyup
% 19 - make zip
% 20 - make zip-data

{\sf\begin{tabular}{rp{4.5in}}

\texttt{make all}&
      Analyze the data, then typeset the main PDF files (\texttt{paper-seb-main.pdf} and \texttt{paper-seb-supplementary-material.pdf}).\\
   \multicolumn{1}{l@{\vdots}}{}&\\

\texttt{make data}&
      Analyze the data, and generate all the data files, the Unix scripts, the CSV, and \LaTeX\ files (including the \LaTeX\ summary of this makefile), etc. This \texttt{make} option runs \texttt{node programs/data.js}, downloads the Git repositories used in the pilot survey, and then analyzes them. Note that downloading all the repositories in a reasonable time needs decent internet bandwidth.\\
   \multicolumn{1}{l@{\vdots}}{}&\\

\texttt{make git-prep}&
      What's on Git that we've lost, or stuff we have got locally but probably don't want on Git, so you can delete it or move it out the way or whatever.\\
   \multicolumn{1}{l@{\vdots}}{}&\\

\texttt{make really-tidyup}&
      More thorough than \texttt{make tidyup} --- remove \emph{all} files that can be recreated. (This will mean next time you run \LaTeX\ you will have to ignore errors as the .aux files are re-created.)\\
\end{tabular}}

