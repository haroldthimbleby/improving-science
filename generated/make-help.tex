% Edit makefile to change the selected make command choices, as follows:
% 1 - make all
% 2 - make archive
% 3 - make check-same
% 4 - make check-update
% 5 - make check-versions
% 6 - make data
% 7 - make expand
% 8 - make help
% 9 - make help-brief
% 10 - make mathematica
% 11 - make mathematica-open
% 12 - make one-file
% 13 - make pdf
% 14 - make push
% 15 - make readme
% 16 - make really-tidyup
% 17 - make tidyup
% 18 - make zip
% 19 - make zip-data

{\sf\begin{tabular}{rp{4.5in}}

\texttt{make all}&
      Analyze the data, then typeset the main PDF files (\texttt{paper-seb-main.pdf} and \texttt{paper-seb-supplementary-material.pdf}).\\
   \multicolumn{1}{l@{\vdots}}{}&\\

\texttt{make data}&
      Analyze the data, and generate all the data files, the Unix scripts, the CSV, and \LaTeX\ files (including the \LaTeX\ summary of this makefile), etc. This \texttt{make} option runs \texttt{node programs/data.js}, downloads the Git repositories used in the pilot survey, and then analyzes them. Note that downloading all the repositories in a reasonable time needs decent internet bandwidth.\\
   \multicolumn{1}{l@{\vdots}}{}&\\

\texttt{make help}&
      Explain how to use \texttt{make}, and list all available options for using \texttt{make}.\\
   \multicolumn{1}{l@{\vdots}}{}&\\

\texttt{make tidyup}&
      Tidyup before doing a Git commit. Remove all easily generated files, and the large Git repositories needed for the pilot survey. Do not remove the main PDFs, or the \LaTeX\ data include files. Do not remove the .aux files, as \LaTeX\ runs much more smoothly with them.\\
\end{tabular}}

